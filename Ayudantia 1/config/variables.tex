% Definiciones de comandos, para reutilizar secuencias frecuentes o ahorrar código
\newcommand{\mytitle}{Ayudantía 1}
\newcommand{\tema}{Límites}
\newcommand{\fecha}{15-03-2024}

\newcommand{\ayudante}{Cristóbal Rojas}
\newcommand{\mailuc}{cristobalrojas@uc.cl}

\newcommand{\facultad}{Facultad de Matemáticas}
\newcommand{\semestre}{Primer Semestre del 2024}

\newcommand{\siglacurso}{MAT1610}
\newcommand{\nombrecurso}{Cálculo I}
\newcommand{\numseccion}{1}
\newcommand{\profesor}{Thomas Führer}
\newcommand{\mailprofesor}{tofuhrer@mat.uc.cl}

\newcommand{\ds}{\displaystyle}

\pagestyle{fancy}
\fancyhf{}
\renewcommand{\headrulewidth}{0pt}
\renewcommand{\footrulewidth}{0.35pt}
\setlength\parindent{0pt}

% Ubicación de figuras
\graphicspath{{./figuras/}}

% Definir color de hipervínculos
\hypersetup{
  colorlinks=false,
  linkbordercolor=0.96 0.60 0.14,   % Código RGB
  urlbordercolor=0.96 0.60 0.14,    % Código RGB
% hidelinks                         % Descomentar esta línea borra los bordes alrededor de hipervínculos
}

% Creación del pie de página
\lfoot{\footnotesize{\fecha \hfill \siglacurso \ -- \mytitle \hfill Página \thepage{} de \pageref{LastPage}}}
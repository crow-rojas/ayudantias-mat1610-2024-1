Para la función $f(x)=\dfrac{\sqrt{x}-\sqrt{2}}{|x-2|}$, responda las siguientes preguntas:

\begin{enumerate}[label=\alph*)]
  \item Determine el valor de $\ds\lim _{x \rightarrow 0^{+}} f(x)$.
  \item ¿Existe el $\ds\lim _{x \rightarrow 2} f(x)$? Justifique su respuesta. En caso afirmativo, ¿cuál es su valor?
\end{enumerate}

\begin{soluciones}
  \subsubsection*{Solución}

  Note que, el dominio de la función $f$ son los valores reales no negativos diferentes de 2 y que

  $$
  |x-2|=\left\{\begin{array}{lll}
  x-2 & \text { si } & x \geq 2 \\
  2-x & \text { si } & x<2
  \end{array}\right.
  $$
  
  entonces, la función $f$ puede escribirse como

  $$
  f(x)=\left\{\begin{array}{ccc}
  \dfrac{\sqrt{x}-\sqrt{2}}{2-x} & \text { si } & 0 \leq x<2 \\[15pt]
  \dfrac{\sqrt{x}-\sqrt{2}}{x-2} & \text { si } & x>2
  \end{array}\right.
  $$

  o, equivalentemente, como

  $$
  f(x)=\left\{\begin{array}{lcc}
  \dfrac{1}{-(\sqrt{x}+\sqrt{2})} & \text { si } & 0 \leq x<2 \\[15pt]
  \dfrac{1}{\sqrt{x}+\sqrt{2}} & \text { si } & x>2
  \end{array}\right.
  $$

  ya que

  $$
  2-x=(\sqrt{2}-\sqrt{x})(\sqrt{2}+\sqrt{x})=-(\sqrt{x}-\sqrt{2})(\sqrt{x}+\sqrt{2})
  $$

  y luego

  $$
  x-2=(\sqrt{x}-\sqrt{2})(\sqrt{x}+\sqrt{2})
  $$

  Así,

  $$
  \begin{aligned}
  \lim _{x \rightarrow 0^{+}} f(x) & =\lim _{x \rightarrow 0^{+}} \frac{1}{-(\sqrt{x}+\sqrt{2})} \\
  & =\frac{\lim _{x \rightarrow 0} 1}{-\left(\lim _{x \rightarrow 0^{+}} \sqrt{x}+\lim _{x \rightarrow 0} \sqrt{2}\right)} \\
  & =\frac{1}{-(0+\sqrt{2})} \\
  & =-\frac{1}{\sqrt{2}} \\
  & =-\frac{\sqrt{2}}{2}
  \end{aligned}
  $$

  Para la parte b) se estudian los límites laterales

  $$
  \begin{aligned}
  \lim _{x \rightarrow 2^{-}} f(x) & =\lim _{x \rightarrow 2^{-}} \frac{1}{-(\sqrt{x}+\sqrt{2})} \\
  & =\frac{\lim _{x \rightarrow 0} 1}{-\left(\lim _{x \rightarrow 2^{-}} \sqrt{x}+\lim _{x \rightarrow 2^{-}} \sqrt{2}\right)} \\
  & =\frac{1}{-(\sqrt{2}+\sqrt{2})} \\
  & =-\frac{1}{2 \sqrt{2}} \\
  & =-\frac{\sqrt{2}}{4}
  \end{aligned}
  $$

  $$
  \begin{aligned}
  \lim _{x \rightarrow 2^{+}} f(x) & =\lim _{x \rightarrow 2^{+}} \frac{1}{\sqrt{x}+\sqrt{2}} \\
  & =\frac{\lim _{x \rightarrow 0} 1}{\lim _{x \rightarrow 2^{+}} \sqrt{x}+\lim _{x \rightarrow 2^{+}} \sqrt{2}} \\
  & =\frac{1}{\sqrt{2}+\sqrt{2}} \\
  & =\frac{1}{2 \sqrt{2}} \\
  & =\frac{\sqrt{2}}{4}
  \end{aligned}
  $$

  Entonces, como $\lim _{x \rightarrow 2^{+}} f(x) \neq \lim _{x \rightarrow 2^{-}} f(x)$, el $\lim _{x \rightarrow 2} f(x)$ no existe.
\end{soluciones}
Determine, si corresponde, la ecuación de la(s) asíntota(s) vertical(es) de la función dada:

\begin{enumerate}[label=\alph*)]
  \item $f(x)=\dfrac{\sqrt{4 x^2+2020}}{3 x-6}$
  \item $f(x)=\dfrac{x^3-x}{x^2-6 x+5}$
  \item $f(x)=\dfrac{-2 e^x}{e^x-5}$
\end{enumerate}

\begin{soluciones}
  \subsubsection*{Solución}

  \begin{enumerate}[label=\alph*)]
    \item $f(x)=\dfrac{\sqrt{4 x^2+2020}}{3 x-6}$
    
    Note que el denominador de $f(x)$ es cero si $x=2$, por lo que la recta de ecuación $x=2$ es una posible asíntota vertical. El numerador es $\sqrt{2036}$, que es mayor que cero, y cuando $x$ tiende a 2 por la derecha $\left(2^{+}\right)$, la expresión $3 x-6$ toma valores positivos que tienden a cero. Entonces,

    $$
    \lim _{x \rightarrow 2^{+}} f(x)=\lim _{x \rightarrow 2^{+}} \frac{\sqrt{4 x^2+2020}}{3 x-6}=+\infty
    $$

    Así, la recta de ecuación $x=2$ es una asíntota vertical de $f$.
    Además, debido a que cuando $x$ tiende a 2 por la izquierda $\left(2^{-}\right)$, la expresión $3 x-6$ toma valores negativos que tienden a cero, se tiene que

    $$
    \lim _{x \rightarrow 2^{-}} f(x)=\lim _{x \rightarrow 2^{-}} \frac{\sqrt{4 x^2+2020}}{3 x-6}=-\infty
    $$

    lo cual, también indica que la recta de ecuación $x=2$ es una asíntota vertical de la función $f$.

    \item $f(x)=\dfrac{x^3-x}{x^2-6 x+5}$
    
    En este caso el denominador de la función racional dada se anula cuando $x=1$ y cuando $x=5$, por lo que las rectas de ecuación $x=1$ y $x=5$ son posibles asíntotas verticales.

    Sin embargo, para el caso en el que $x=1$, se tiene que

    $$
    \begin{aligned}
    \lim _{x \rightarrow 1} f(x) & =\lim _{x \rightarrow 1} \frac{x(x-1)(x+1)}{(x-1)(x-5)} \\
    & =\lim _{x \rightarrow 1} \frac{(x-1)}{x-1} \frac{x(x+1)}{(x-5)} \\
    & =\lim _{x \rightarrow 1} \frac{(x-1)}{(x-1)} \lim _{x \rightarrow 1} \frac{x(x+1)}{(x-5)} \\
    & =1 \cdot\left(\frac{2}{-4}\right) \\
    & =-\frac{1}{2}
    \end{aligned}
    $$

    es decir, el límite es finito, por lo que la recta de ecuación $x=1$ no corresponde a una asíntota vertical de la función $f$.

    Para el caso de $x=5$,

    $$
    \begin{aligned}
    \lim _{x \rightarrow 5^{+}} f(x) & =\lim _{x \rightarrow 5^{+}} \frac{x(x-1)(x+1)}{(x-1)(x-5)} \\
    & =\lim _{x \rightarrow 5^{+}} \frac{(x-1)}{(x-1)} \lim _{x \rightarrow 5^{+}} \frac{x(x+1)}{x-5} \\
    & =1 \cdot \lim _{x \rightarrow 5^{+}} \frac{x(x+1)}{(x-5)} \\
    & =\lim _{x \rightarrow 5^{+}} \frac{x(x+1)}{(x-5)} \\
    & =\infty
    \end{aligned}
    $$

    ya que cuando $x$ tiende a 5 por la derecha $\left(5^{+}\right)$, la expresión $x(x+1)$ tiende a 30 , que es mayor que cero, y la expresión $x-5$ toma valores positivos que tienden cero. Entonces, la recta de ecuación $x=5$ es una asíntota vertical de $f(x)$.

    También, observe que

    $$
    \begin{aligned}
    \lim _{x \rightarrow 5^{-}} f(x) & =\lim _{x \rightarrow 5^{-}} \frac{x(x-1)(x+1)}{(x-1)(x-5)} \\
    & =\lim _{x \rightarrow 5^{+}} \frac{(x-1)}{(x-1)} \lim _{x \rightarrow 5^{+}} \frac{x(x+1)}{x-5} \\
    & =1 \cdot \lim _{x \rightarrow 5^{-}} \frac{x(x+1)}{(x-5)} \\
    & =\lim _{x \rightarrow 5^{-}} \frac{x(x+1)}{(x-5)} \\
    & =-\infty
    \end{aligned}
    $$

    ya que cuando $x$ tiende a 5 por la izquierda ($5^-$), la expresión $x(x+1)$ tiende a 30 , que es mayor que cero, y la expresión $x-5$ toma valores negativos que tienden cero.
    El hecho de que el $\lim _{x \rightarrow 5^{-}} f(x)=-\infty$ también indica que la recta de ecuación $x=5$ es una asíntota vertical de $f$.

    \item $f(x)=\dfrac{-2 e^x}{e^x-5}$
    
    Para esta función el denominador de $f(x)$ es cero si $x=\ln (5)$. El numerador es $-2 e^{\ln (5)}=-2 \cdot 5=-10$, que es menor que cero, entonces,

    $$
    \lim _{x \rightarrow\left((\ln (5))^{+}\right.} f(x)=\lim _{x \rightarrow\left((\ln (5))^{+}\right.} \frac{-2 e^x}{e^x-5}=-\infty
    $$

    ya que cuando $x$ tiende a $\ln (5)$ por la derecha $\left((\ln (5))^{+}\right)$, la expresión $e^x-5$ toma valores positivos que tienden a cero. Así, la recta de ecuacón $x=\ln (5)$ es una asíntota vertical de $f$.
    Además,

    $$
    \lim _{x \rightarrow\left((\ln (5))^{-}\right.} f(x)=\lim _{x \rightarrow\left((\ln (5))^{-}\right.} \frac{-2 e^x}{e^x-5}=\infty
    $$

    ello debido a que cuando $x$ tiende a $\ln (5)$ por la izquierda $\left((\ln (5))^{-}\right)$, la expresión $e^x-5$ toma valores negativos que tienden a cero. Esto también indica que $x=\ln (5)$ es una asíntota vertical de $f$.
  \end{enumerate}
\end{soluciones}
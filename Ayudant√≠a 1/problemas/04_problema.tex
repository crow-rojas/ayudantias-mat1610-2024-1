Demuestre, usando la definición, que

$$\lim_{x \rightarrow-1} \frac{-5+3 x}{2}=-4
$$

\begin{soluciones}
  \subsubsection*{Solución}

  Sea $\varepsilon$ un número positivo dado. Se debe determinar un número positivo $\delta$ tal que:

  $$
  \text { si }|x-(-1)|<\delta \text { entonces }\left|\frac{-5+3 x}{2}-(-4)\right|<\varepsilon
  $$

  o, equivalentemente,

  $$
  \text { si }|x+1|<\delta \text { entonces }\left|\frac{-5+3 x}{2}+4\right|<\varepsilon
  $$

  Note que,

  $$
  \begin{aligned}
  \left|\frac{-5+3 x}{2}+4\right| & =\left|\frac{-5+3 x+8}{2}\right| \\
  & =\left|\frac{3 x+3}{2}\right| \\
  & =\left|\frac{3}{2}(x+1)\right| \\
  & =\frac{3}{2}|x+1|
  \end{aligned}
  $$

  y, si $|x+1|<\delta$, entonces $\frac{3}{2}|(x+1)|<\frac{3}{2} \delta$. Es decir, que:

  $$
  \text { si }|x+1|<\delta \text { entonces, }\left|\frac{-5+3 x}{2}+4\right|=\frac{3}{2}|x+1|<\frac{3}{2} \delta
  $$

  Así, para garantizar que $\left|\frac{-5+3 x}{2}+4\right|<\epsilon$, basta garantizar que $\frac{3}{2} \delta \leq \varepsilon$. En particular, basta considerar $\delta=\frac{2}{3} \varepsilon$. \medbreak

  Para la verificación, observe que, para $\varepsilon>0$ dado, si se toma $\delta=\frac{2}{3} \varepsilon$
  $$
  \begin{aligned}
  |x+1|<\delta & \Longrightarrow|x+1|<\frac{2}{3} \varepsilon \\
  & \Longrightarrow \frac{3}{2}|x+1|<\frac{3}{2} \frac{2}{3} \varepsilon \\
  & \Longrightarrow \underbrace{\frac{3}{2}|x+1|}_{\downarrow}<\varepsilon \\
  & \Longrightarrow\left|\frac{-5+3 x}{2}+4\right|<\varepsilon
  \end{aligned}
  $$
  lo cual demuestra lo requerido.
\end{soluciones}